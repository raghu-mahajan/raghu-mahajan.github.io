\documentclass{article}
\usepackage{amsmath}
\usepackage{amssymb}

\renewcommand{\d}{\text{d}}
\renewcommand{\i}{\text{i}}

\begin{document}

Perhaps the simplest quantum systems with a discrete spectrum are (a) a particle in a box and (b) a particle on a ring.
The presence of Dirichlet walls in the particle in a box problem presents some difficulties and so,  for that reason,   the particle on a ring is the simplest system with a discrete spectrum that is amenable to a path integral treatment.

Let $x$ be the coordinate on the ring,  which is identified with period $2\pi R$.
In the Hamiltonian formalism,  we can simply find the eigenvalues and eigenvectors of the system.
\begin{equation}
\begin{aligned}
H &= - \frac{1}{2} \frac{\d^2}{\d x^2} \\
\psi_n(x) &=  \exp\left(\i  \frac{n}{R} x \right) \frac{1}{\sqrt{2\pi R}} \,,\quad n \in \mathbb{Z} \\
E_n &= \frac{1}{2} \,  \frac{n^2}{R^2}\,.
\end{aligned}
\end{equation}
The Hilbert space is infinite dimensional, there is a unique ground state with zero energy,  and all other energy levels are doubly degenerate.

We would like to compute the thermal partition function of this system.
Since we know the spectrum,  we can write down the answer
\begin{align}
Z(\beta) = \sum_n e^{-\beta E_n} = \sum_{n\in \mathbb{Z}} \exp \left( 
- \beta \, \frac{n^2}{2R^2}
\right) \, .
\label{zbetaparticlering}
\end{align}

\end{document}